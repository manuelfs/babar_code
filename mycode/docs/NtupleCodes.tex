\documentclass[10pt,twocolumn]{article}
%\documentclass[6pt]{article}
\usepackage{lscape}
\usepackage{amsmath}
\usepackage{colortbl}
\usepackage[left=2cm,top=1cm,right=3cm,nohead,nofoot]{geometry}
\pagestyle{empty}
\input{babarsym.tex}

\newcommand{\Red} [1]  {\textcolor{red}{#1}}

\begin{document} 



\begin{tabular}{r l}\\ \hline\hline
\multicolumn{2}{c}{\bf MCType}\\ \hline
0 & Combinatoric\\
1 & $B\to\Dz e\nu$\\
2 & $B\to\Dstarz e\nu$\\
3 & $B\to\Dz\mu\nu$\\
4 & $B\to\Dstarz\mu\nu$\\
5 & $B\to\Dz\tau\nu$\\
6 & $B\to\Dstarz\tau\nu$\\
7 & $B\to\Dp e\nu$\\
8 & $B\to\Dstarp e\nu$\\
9 & $B\to\Dp\mu\nu$\\
10 & $B\to\Dstarp\mu\nu$\\
11 & $B\to\Dp\tau\nu$\\
12 & $B\to\Dstarp\tau\nu$\\
13 & $B\to D^{(*)}\pi\ell\nu$\\
14 & $B\to D^{**}\ell\nu$\\ \hline\hline
\end{tabular}

\noindent In this categorization, ``combinatoric'' means none of
the other numbered modes. Each of the other modes requires
the tag $B$ to decay hadronically, so that any event with
two semileptonic $B$ decays is also marked with MCType = 0.

\begin{tabular}{r l}\\ \hline\hline
\multicolumn{2}{c}{\bf candType}\\ \hline
1 & $\Dz$\\
2 & $\Dstarz$\\
3 & $\Dp$\\
4 & $\Dstarp$\\
5 & $\Dz\piz$\\
6 & $\Dstarz\piz$\\
7 & $\Dp\piz$\\
8 & $\Dstarp\piz$\\ \hline\hline
\end{tabular}

\noindent The ntuple only uses candType $\le$ 4, and
these channels include both the signal box and the
$D^{**}$ control sample. When we go from the ntuples
to the fit samples, the signal and control samples
are split, and candTypes 5--8 are introduced.


\begin{tabular}{r l}\\ \hline\hline
\multicolumn{2}{c}{\bf MCCombmode}\\ \hline
1 & $\Ds\to\tau X$\\
2 & $\tau$ not from \Ds\\
3 & $ee$\\
4 & $\mu\mu$\\
5 & $e\mu$\\
6 & $D \to e$\\
7 & $D \to \mu$\\
8 & $e$ not coming from $D$ or $\tau$\\ 
9 & $\mu$ not coming from $D$ or $\tau$\\ 
10 & More than 3 leptons not coming from $\tau$\\ 
11 & There is a $K_L^0$\\ 
12 & Everything else\\ 
\hline\hline
\end{tabular}

\noindent These codes are ranked, so that, should an event
fall into multiple categories, the lowest numbered flag is
selected.


\begin{tabular}{r l | l}\\ \hline\hline
 & {\bf \Dz Decay} & {\bf \Dp Decay}\\ \hline
1 & $K\pi$ & $K\pi\pi$\\
2 & $K\pi\piz$ & $K\pi\pi\piz$\\
3 & $K3\pi$ & $\KS\pi$\\
4 & $\KS\pi\pi$ & $\KS\pi\piz$\\
5 & $\KS\pi\pi\piz$ & $\KS 3\pi$\\
6 & $\KS\KS$ (not used) & $KK\pi$ (not used)\\ 
7 & $\KS\piz$ & $\KS K$\\
8 & $\pi\pi$ (not used)& ---\\
9 & $K K$ & ---\\ \hline\hline
 & {\bf \Dstarz Decay} & {\bf \Dstarp Decay}\\ \hline
1 & $\Dz\piz$ & $\Dz\pi$\\
2 & $\Dz\gamma$ & $\Dp\piz$\\
3 & --- & $\Dp\gamma$ (not used) \\\hline\hline
\end{tabular}

\noindent {\bf MCSubmode} uses the same decay codes, and encodes
\Dstar decays as $100\times$DstarType $+$DType. Note that
\Dstar mode 3 is only used in MCSubmode, as it is not
reconstructed.



\begin{tabular}{l l r}\\ \hline\hline
Channel          & Source  & pdf name\\ \hline
$\Dstarz\ellm$   & $\Dstarz\taum\nutb$ signal           & rds[3]   \\
                 & $\Dz\taum\nutb$ signal feed-up       & rds[9] \\
                 & $\Dstarz\ellm\nulb$ normalization    & rdc[0]  \\
                 & $\Dz\ellm\nulb$ feed-up              & rds[8]   \\
                 & $D^{**}(\ellm/\taum)\nub$ feed-down  & rds[6]   \\
                 & Charge-crossfeed                     & comb[3] \\
                 & Combinatorial background             & comb[0] \\ \hline
$\Dz\ellm$       & $\Dz\taum\nutb$ signal               & rds[5] \\
                 & $\Dstarz\taum\nutb$ signal feed-down & rds[4] \\
                 & $\Dz\ellm\nulb$ normalization        & rdc[1] \\
                 & $\Dstarz\ellm\nulb$ feed-down        & rds[8] \\
                 & $D^{**}\ellm\nulb$ feed-down         & rds[7] \\
                 & Charge-crossfeed                     & comb[2] \\
                 & Combinatorial background             & comb[0] \\ \hline
$\Dstarp\ellm$   & $\Dstarp\taum\nutb$ signal           & rds[13] \\
                 & $\Dp\taum\nutb$ signal feed-up       & rds[19] \\
                 & $\Dstarp\ellm\nulb$ normalization    & rdc[2] \\
                 & $\Dp\ellm\nulb$ feed-up              & rds[18]   \\
                 & $D^{**}\ellm\nulb$ feed-down         & rds[16] \\
                 & Charge-crossfeed                     & comb[3] \\
                 & Combinatorial background             & comb[0] \\ \hline
$\Dp\ellm$       & $\Dp\taum\nutb$ signal               & rds[15] \\
                 & $\Dstarp\taum\nutb$ signal feed-down & rds[14] \\
                 & $\Dp\ellm\nulb$ normalization        & rdc[3] \\
                 & $\Dstarp\ellm\nulb$ feed-down        & rdc[9] \\
                 & $D^{**}\ellm\nulb$ feed-down         & rds[17] \\
                 & Charge-crossfeed                     & comb[2] \\
                 & Combinatorial background             & comb[0] \\ \hline\hline
\end{tabular}

\begin{tabular}{l l r}\\ \hline\hline
Channel          & Source  & pdf name\\ \hline
$\Dstarz\piz\ellm$ & $D^{**}\ellm\nulb$                & rdc[5] \\
              & $\Dstarz\ellm\nulb$ feed-up            & rds[24] \\
              & $\Dz\ellm\nulb$ feed-up                & rds[25] \\
              & Charge-crossfeed                       & comb[4] \\
              & Combinatorial background               & comb[1] \\ \hline
$\Dz\piz\ellm$     & $D^{**}\ellm\nulb$                & rdc[4] \\
              & $\Dstarz\ellm\nulb$ feed-up            & rds[21]  \\
              & $\Dz\ellm\nulb$ feed-up                & rds[22] \\
              & Charge-crossfeed                       & comb[4] \\
              & Combinatorial background               & comb[1] \\ \hline
$\Dstarp\piz\ellm$ & $D^{**}\ellm\nulb$                & rdc[7] \\
              & $\Dstarp\ellm\nulb$ feed-up            & rds[30] \\
              & $\Dp\ellm\nulb$ feed-up                & rds[31]   \\
              & Charge-crossfeed                       & comb[4]   \\
              & Combinatorial background               & comb[1] \\ \hline
$\Dp\piz\ellm$     & $D^{**}\ellm\nulb$                & rdc[6] \\
              & $\Dstarp\ellm\nulb$ feed-up            & rds[27] \\
              & $\Dp\ellm\nulb$ feed-up                & rds[28] \\
              & Charge-crossfeed                       & comb[4] \\
              & Combinatorial background               & comb[1] \\ \hline\hline
\end{tabular}





\end{document}
