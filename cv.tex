%\documentclass[]{article}

\documentclass[12pt,letterpaper]{MACPcv}
\usepackage{hyperref}
\input{pubboard/babarsym.tex}

% Input the BaBar symbols file
%\input pubboard/babarsym

% KKK0 symbols 
\def\KKKz      {\ensuremath{K^+ K^- K^0}\xspace}
\def\KKKs      {\ensuremath{K^+ K^- \KS}\xspace}
\def\KKKl      {\ensuremath{K^+ K^- \KL}\xspace}
\def\KKKspm      {\ensuremath{K^+ K^- {\KS}(\pip\pim)}\xspace}
\def\KKKszz      {\ensuremath{K^+ K^- {\KS}(\piz\piz)}\xspace}
\def\KKsKs      {\ensuremath{K^+ \KS \KS}\xspace}
\def\phiKs     {\ensuremath{\phi \KS}\xspace}
\def\phiKz     {\ensuremath{\phi K^0}\xspace}
\def\splot{\ensuremath{_s{\cal P}lot}\xspace}
\def\splots{\ensuremath{_s{\cal P}lot}s\xspace}
\def\mKK   {\ensuremath{m_{\Kp\Km}}\xspace}
\def\mKpKz {\ensuremath{m_{\Kp\Kz}}\xspace}
\def\mKmKz {\ensuremath{m_{\Km\Kz}}\xspace}
\def\betaeff {\ensuremath{\beta_{\mathit{eff}}}\xspace}
\def\Acp     {\ensuremath{{A}_{\CP}}\xspace}
\def\cosH    {\ensuremath{\cos \theta_H}\xspace}
\def\cosT    {\ensuremath{\cos \theta_{\rm T}}\xspace}
\def\bra     {\ensuremath{\left <}}
\def\ket     {\ensuremath{\right >}}
\def\spk{\ensuremath{S_{\phi K}}\xspace}
\def\cpk{\ensuremath{C_{\phi K}}\xspace}

\def\fisher {\ensuremath{\mathcal{F}}\xspace}
\def\LowMass {Low-mass\xspace}
\def\HighMass {High-mass\xspace}
\def\hjphi {\ensuremath{\phi(1020)\xspace}}
\def\fzone {\ensuremath{f_0(980)\xspace}}
\def\hjX {\ensuremath{X(1550)\xspace}}
\newcommand{\bkkkboth}{\ensuremath{ \Bpm \rightarrow \Kpm\Kmp\Kpm}\xspace}
\newcommand{\bkksks}{\ensuremath{ \Bpm \rightarrow \KS\KS\Kpm}\xspace}
\newcommand{\bkkks}{\ensuremath{\Bz \rightarrow \Kp\Km\KS}\xspace}

\name{Brian Lindquist}


\begin{document}

\begin{cv}




% The body of the paper starts here
\begin{personal}
%\section{Contact Information}


Work address:  & SLAC National Accelerator Laboratory \\
               & 2575 Sand Hill Rd. \\
               & Mail Stop 95   \\
               & Menlo Park, CA 94025  \\
Phone:         &   (650) 926-3494        \\
E-mail address:  & lindquis@slac.stanford.edu  \\
\end{personal}


\section{Education}

\begin{datelist}
\item[2011 (anticipated)] \dotfill \textbf{Ph.D. - Physics} \\
Stanford University, Stanford CA.
\item[May 2005] \dotfill \textbf{B.S. - Physics, German (double major)} \\
Iowa State University, Ames IA.
%--How about the end-of-the-year Physics department awards? Should I list here?
%--What about Phi Beta Kappa?
\end{datelist}

\section{Honors}
B.S. with Distinction, 2005 \\
Outstanding Physics Award/Hammer Scholarship, 2004. \\
Outstanding Physics Award/Schirber Scholarship, 2003. \\
Phi Beta Kappa, 2003. \\



\section{Research}

\begin{Simplelist}

\item
{\bf {\boldmath \CP} and Dalitz plot analyses of {\boldmath \bkkks, \bkkkboth}, and {\boldmath \bkksks}}

Dissertation research. Performed a measurement of \CP violation in a time-dependent
Dalitz plot analysis of \bkkks.  This included an improved measurement of the
\CP-violating angle $\beta_{eff}$ in $B^0\to\phi\KS$, which is a theoretically clean
probe of physics beyond the Standard Model.  Carried out \CP and amplitude analyses of the 
$\bkkks,\bkkkboth,$ and $\bkksks$ Dalitz plots, including the first such analysis of \bkksks. 
By analyzing the Dalitz plots, was able to measure the S-wave and P-wave
components, which is important for reducing the systematic error on the \CP violation 
measurement in $\bkkks$.  Also found evidence, with the aid of angular analysis, that
a Dalitz plot structure previously thought to be a single scalar resonance, actually
consists of multiple overlapping resonances, including a tensor component.  %Should I remove,
%or rewrite, this last sentence?  Should ask Aaron.
\\
\item
{\bf Development of cosmic telescope for testing pixel sensors for ATLAS}

Helped design and construct a cosmic telescope for use in testing new pixel 
sensors, such as the 3D pixel sensors
being designed for possible use in an ATLAS upgrade. Developed simulation code
for the telescope, and also worked on event reconstruction and physics analysis
of the sensors. Measured sensor efficiency with
the telescope.  Also participated in a test beam at CERN, at which both 3D and
planar pixel sensors were tested.   

\item
{\bf Measurement of the PEP-II interaction point luminous region with {\boldmath 
$e^+e^-\rightarrow\mu^+\mu^-$} events}

Worked on an analysis that used $e^+e^-\rightarrow\mu^+\mu^-$ events measured by the BaBar 
detector to reconstruct the transverse size of the luminous region in the PEP-II collider,
as a function of the z-position. This measurement allows one to 
extrapolate the value of $\beta^{*}_y$, a parameter that describes how
focused the beams are at the interaction point.




%-Eta Eta Ks -- Aaron says kind of optional, maybe don't mention

%-Maybe very brief mention of Stanford rotations. --Aaron says probably don't mention these

\end{Simplelist}



%\section{Selected Papers}
\begin{publications}

  \begin{Simplelist}

\item
{\bf Interaction-point phase-space characterization using single-beam and luminous-region measurements at PEP-II}
\\{}W.~Kozanecki {\it et al}
\\{}Nucl.\ Instr.\ and\ Meth.\ A {\bf 607} (2009), p. 293.


\item
{\bf Measurement of {\boldmath $\CP$}-Violating Asymmetries in the {\boldmath $\Bz\to\Kp\Km\KS$} Dalitz Plot}
\\{} B.~Aubert {\it et al} [\babar\ Collaboration]
\\{} arXiv:0808.0700 [hep-ex]
\\{} {\it Submitted to the 34th International Conference on High
Energy Physics (ICHEP 2008), Philadelphia, PA, 30 Jul - 5 Aug 2008}.
%[arXiv:0808.0700v2].



\item
{\bf {\boldmath \bkksks} Dalitz plot analysis with Run1-6 dataset}
\\ {}\babar\ Analysis Document \#2338 [internal document], July 14, 2011.

\item
{\bf {\boldmath \bkkkboth} Dalitz plot analysis with Run1-6 dataset}
\\ {}\babar\ Analysis Document \#2116 [internal document], July 14, 2011.


\item
{\bf {\boldmath \bkkks} Dalitz plot analysis with Run1-6 R24 dataset}
\\ {}\babar\ Analysis Document \#2383 [internal document], June 7, 2011.


\item
{\bf Charmless {\boldmath \B} Decays to {\boldmath $\eta\eta\KS$} and {\boldmath $\eta\piz\KS$}}
\\ {}\babar\ Analysis Document \#1730 [internal document], Oct 26, 2008.


\item
{\bf {\boldmath \bkkks} Dalitz plot analysis with full dataset}
\\ {}\babar\ Analysis Document \#1991 [internal document], July 28, 2008.


  \end{Simplelist}

\end{publications}



\section{Presentations}

\begin{Simplelist}

\item {\it Charmless Hadronic B Decays with BABAR} at the 2011 DPF Meeting, 
Providence, RI, August 2011.

\item  {\it Dalitz Plots},  SLAC Association for Student Seminars, Nov 2010.

\item {\it Study of the Decays $B^0 \to K^+K^-K^0_S$ and $B^+ \to K^+K^-K^+$} at the 2009 APS April Meeting, 
Denver, CO, May 2009.

%(or should I use the title according to the PPT Title, which was ``$B^0 \to K^+K^-K^0_S$ Dalitz Analysis
%at BaBar'')

\end{Simplelist}




\section{Summer Schools Attended}

\begin{Simplelist}

\item {\it Fifth CERN-Fermilab Hadron Collider Physics Summer School}, Fermilab, August, 2010.

\item {\it XXXVII SLAC Summer Institute, Revolutions on the Horizon: A Decade of New Experiments}, SLAC, 
August, 2009.

\item {\it XXXV SLAC Summer Institute, Dark Matter: From the Cosmos to the Laboratory}, SLAC, July-August, 2007.

\end{Simplelist}





\section{Teaching Experience}
\begin{datelist}

\item[Spring 2008] Teaching Assistant, Stanford University. Physics 134: Advanced Topics in Quantum Mechanics.

\item[Winter 2006] Teaching Assistant, Stanford University. Physics 23:  Electricity and Optics.

\item[Autumn 2005] Teaching Assistant, Stanford University. Physics 21:  Mechanics and Heat.

\end{datelist}



\section{References}



Prof. Aaron Roodman, SLAC, roodman@slac.stanford.edu

Prof. Vera L\"{u}th, SLAC, luth@slac.stanford.edu

Prof. Jacques Chauveau, LPNHE, Paris Univ. 6 and 7, chauveau@in2p3.fr




\end{cv}


\end{document}
