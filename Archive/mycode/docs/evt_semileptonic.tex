\documentclass[6pt]{article}
\usepackage{lscape}
\usepackage{amsmath}
\usepackage{colortbl}
\usepackage[left=2cm,top=1cm,right=3cm,nohead,nofoot]{geometry}
\pagestyle{empty}
%%%%%   Standard symbols for use in BABAR papers and BAD Notes
%%%%%
%%%%%   Revised   05/22/01 P. Dauncey     Added \hepex, etc. and clean up a bit
%%%%%   Revised   12/07/00 D. Hitlin      Added features of D. Kirkby's HEP.sty
%%%%%   Revised   07/13/00 R. Waldi       Corrected \Kbar, \Bbar ... macros
%%%%%   Revised   07/13/00 D. MacFarlane  Replaced incorrect \chic1 symbols
%%%%%   Revised   07/05/00 P. Dauncey     Added \mes, \mec, removed \O
%%%%%   Revised   07/04/00 D. MacFarlane  Added scalable version of BABAR
%%%%%   Revised   07/01/00 D. MacFarlane
%%%%%   Revised   06/21/00 D. Hitlin
%%%%%   Original  06/10/00 D. Hitlin
%%%%%   Revision of TDR and Physics Book symbol file
%%%%%

\RequirePackage{xspace}

%%%%%%%%%%%%%%%%%%%% BABAR ... THE NAME OF THE COLLABORATION %%%%

% Huge boldface
\def\hbabar{\mbox{{\huge\bf\sl B}\hspace{-0.1em}{\LARGE\bf\sl A}\hspace{-0.03em}{\huge\bf\sl B}\hspace{-0.1em}{\LARGE\bf\sl A\hspace{-0.03em}R}}}
% LARGE
\def\Lbabar{\mbox{{\LARGE\sl B}\hspace{-0.15em}{\Large\sl A}\hspace{-0.07em}{\LARGE\sl B}\hspace{-0.15em}{\Large\sl A\hspace{-0.02em}R}}}
% Large
\def\lbabar{\mbox{{\large\sl B}\hspace{-0.4em} {\normalsize\sl A}\hspace{-0.03em}{\large\sl B}\hspace{-0.4em} {\normalsize\sl A\hspace{-0.02em}R}}}
% normal size
%\def\babar{\mbox{\sl B\hspace{-0.4em} {\small\sl A}\hspace{-0.37em} \sl B\hspace{-0.4em} {\small\sl A\hspace{-0.02em}R}}}
% replace normalsize with scalable version       dbm 7/4/00
\usepackage{relsize}
\def\babar{\mbox{\slshape B\kern-0.1em{\smaller A}\kern-0.1em
    B\kern-0.1em{\smaller A\kern-0.2em R}}}

%%%%%%%%%%%%%%%%%%%%%%%%%%%%%%%%%%%%%%%%%%%%%%%
%%%%%%%%%%%%%%%%%   LEPTONS   %%%%%%%%%%%%%%%%%
%%%%%%%%%%%%%%%%%%%%%%%%%%%%%%%%%%%%%%%%%%%%%%%
\let\emi\en
\def\electron   {\ensuremath{e}\xspace}
\def\en         {\ensuremath{e^-}\xspace}   % electron negative (\em is taken)
\def\ep         {\ensuremath{e^+}\xspace}
\def\epm        {\ensuremath{e^\pm}\xspace} 
\def\epem       {\ensuremath{e^+e^-}\xspace}
\def\ee         {\ensuremath{e^-e^-}\xspace}

\def\mmu        {\ensuremath{\mu}\xspace}
\def\mup        {\ensuremath{\mu^+}\xspace}
\def\mun        {\ensuremath{\mu^-}\xspace} % muon negative (\mum is taken)
\def\mumu       {\ensuremath{\mu^+\mu^-}\xspace}
\def\mtau       {\ensuremath{\tau}\xspace}

\def\taup       {\ensuremath{\tau^+}\xspace}
\def\taum       {\ensuremath{\tau^-}\xspace}
\def\tautau     {\ensuremath{\tau^+\tau^-}\xspace}

\def\ellm       {\ensuremath{\ell^-}\xspace}
\def\ellp       {\ensuremath{\ell^+}\xspace}
\def\ellell     {\ensuremath{\ell^+ \ell^-}\xspace}

\def\nub        {\ensuremath{\overline{\nu}}\xspace}
\def\nunub      {\ensuremath{\nu{\overline{\nu}}}\xspace}
\def\nub        {\ensuremath{\overline{\nu}}\xspace}
\def\nunub      {\ensuremath{\nu{\overline{\nu}}}\xspace}
\def\nue        {\ensuremath{\nu_e}\xspace}
\def\nueb       {\ensuremath{\nub_e}\xspace}
\def\nuenueb    {\ensuremath{\nue\nueb}\xspace}
\def\num        {\ensuremath{\nu_\mu}\xspace}
\def\numb       {\ensuremath{\nub_\mu}\xspace}
\def\numnumb    {\ensuremath{\num\numb}\xspace}
\def\nut        {\ensuremath{\nu_\tau}\xspace}
\def\nutb       {\ensuremath{\nub_\tau}\xspace}
\def\nutnutb    {\ensuremath{\nut\nutb}\xspace}
\def\nul        {\ensuremath{\nu_\ell}\xspace}
\def\nulb       {\ensuremath{\nub_\ell}\xspace}
\def\nulnulb    {\ensuremath{\nul\nulb}\xspace}

%%%%%%%%%%%%%%%%%%%%%%%%%%%%%%%%%%%%%%%%%%%%%%%%%%
%%%%%%%%%%%%%%%%%%  PHOTONS  %%%%%%%%%%%%%%%%%%%%%
%%%%%%%%%%%%%%%%%%%%%%%%%%%%%%%%%%%%%%%%%%%%%%%%%%

\def\g     {\ensuremath{\gamma}\xspace}
\def\gaga  {\ensuremath{\gamma\gamma}\xspace}  %% changed from \gg, which is >>
\def\ggstar{\ensuremath{\gamma\gamma^*}\xspace}

\def\ega    {\ensuremath{e\gamma}\xspace}
\def\game   {\ensuremath{\gamma e^-}\xspace}
\def\epemg  {\ensuremath{e^+e^-\gamma}\xspace}

%%%%%%%%%%%%%%%%%%%%%%%%%%%%%%%%%%%%%%%%
%%%%  Other GAUGE BOSONS  %%%%%%%%%%%%%%
%%%%%%%%%%%%%%%%%%%%%%%%%%%%%%%%%%%%%%%%

\def\H      {\ensuremath{H^0}\xspace}
\def\Hp     {\ensuremath{H^+}\xspace}
\def\Hm     {\ensuremath{H^-}\xspace}
\def\Hpm    {\ensuremath{H^\pm}\xspace}
\def\W      {\ensuremath{W}\xspace}
\def\Wp     {\ensuremath{W^+}\xspace}
\def\Wm     {\ensuremath{W^-}\xspace}
\def\Wpm    {\ensuremath{W^\pm}\xspace}
\def\Z      {\ensuremath{Z^0}\xspace}

%%%%%%%%%%%%%%%%%%%%%%%%%%%%%%%%%%%%%%%%%%%%%%%%%%
%%%%%%%%%%%%%%%%%%   QUARKS   %%%%%%%%%%%%%%%%%%%%
%%%%%%%%%%%%%%%%%%%%%%%%%%%%%%%%%%%%%%%%%%%%%%%%%%

\def\q     {\ensuremath{q}\xspace}
\def\qbar  {\ensuremath{\overline q}\xspace}
\def\qqbar {\ensuremath{q\overline q}\xspace}
\def\u     {\ensuremath{u}\xspace}
\def\ubar  {\ensuremath{\overline u}\xspace}
\def\uubar {\ensuremath{u\overline u}\xspace}
\def\d     {\ensuremath{d}\xspace}
\def\dbar  {\ensuremath{\overline d}\xspace}
\def\ddbar {\ensuremath{d\overline d}\xspace}
\def\s     {\ensuremath{s}\xspace}
\def\sbar  {\ensuremath{\overline s}\xspace}
\def\ssbar {\ensuremath{s\overline s}\xspace}
\def\c     {\ensuremath{c}\xspace}
\def\cbar  {\ensuremath{\overline c}\xspace}
\def\ccbar {\ensuremath{c\overline c}\xspace}
\def\b     {\ensuremath{b}\xspace}
\def\bbar  {\ensuremath{\overline b}\xspace}
\def\bbbar {\ensuremath{b\overline b}\xspace}
\def\t     {\ensuremath{t}\xspace}
\def\tbar  {\ensuremath{\overline t}\xspace}
\def\tbar  {\ensuremath{\overline t}\xspace}
\def\ttbar {\ensuremath{t\overline t}\xspace}


%%%%%%%%%%%%%%%%%%%%%%%%%%%%%%%%%%%%%%%%%%%%%%%%%%
%%%%%%%%%%%%%%%%%% LIGHT MESONS  %%%%%%%%%%%%%%%%%
%%%%%%%%%%%%%%%%%%%%%%%%%%%%%%%%%%%%%%%%%%%%%%%%%%

\def\piz   {\ensuremath{\pi^0}\xspace}
\def\pizs  {\ensuremath{\pi^0\mbox\,\rm{s}}\xspace}
\def\ppz   {\ensuremath{\pi^0\pi^0}\xspace}
\def\pip   {\ensuremath{\pi^+}\xspace}
\def\pim   {\ensuremath{\pi^-}\xspace}
\def\pipi  {\ensuremath{\pi^+\pi^-}\xspace}
\def\pipm  {\ensuremath{\pi^\pm}\xspace}
\def\pimp  {\ensuremath{\pi^\mp}\xspace}

\def\kaon  {\ensuremath{K}\xspace}
%%% do NOT use ensuremath here
\def\Kbar  {\kern 0.2em\overline{\kern -0.2em K}{}\xspace}
\def\Kb    {\ensuremath{\Kbar}\xspace}
\def\Kz    {\ensuremath{K^0}\xspace}
\def\Kzb   {\ensuremath{\Kbar^0}\xspace}
\def\KzKzb {\ensuremath{\Kz \kern -0.16em \Kzb}\xspace}
\def\Kp    {\ensuremath{K^+}\xspace}
\def\Km    {\ensuremath{K^-}\xspace}
\def\Kpm   {\ensuremath{K^\pm}\xspace}
\def\Kmp   {\ensuremath{K^\mp}\xspace}
\def\KpKm  {\ensuremath{\Kp \kern -0.16em \Km}\xspace}
\def\KS    {\ensuremath{K^0_{\scriptscriptstyle S}}\xspace} 
\def\KL    {\ensuremath{K^0_{\scriptscriptstyle L}}\xspace} 
\def\Kstarz  {\ensuremath{K^{*0}}\xspace}
\def\Kstarzb {\ensuremath{\Kbar^{*0}}\xspace}
\def\Kstar   {\ensuremath{K^*}\xspace}
\def\Kstarb  {\ensuremath{\Kbar^*}\xspace}
\def\Kstarp  {\ensuremath{K^{*+}}\xspace}
\def\Kstarm  {\ensuremath{K^{*-}}\xspace}
\def\Kstarpm {\ensuremath{K^{*\pm}}\xspace}
\def\Kstarmp {\ensuremath{K^{*\mp}}\xspace}

\newcommand{\etapr}{\ensuremath{\eta^{\prime}}\xspace}


%%%%%%%%%%%%%%%%%%%%%%%%%%%%%%%%%%%%%%%%%%%%%%%%%%
%%%%%%%%%%%%%%%%%% HEAVY MESONS  %%%%%%%%%%%%%%%%%
%%%%%%%%%%%%%%%%%%%%%%%%%%%%%%%%%%%%%%%%%%%%%%%%%%

%%% do NOT use ensuremath here
\def\Dbar    {\kern 0.2em\overline{\kern -0.2em D}{}\xspace}
\def\Db      {\ensuremath{\Dbar}\xspace}
\def\Dz      {\ensuremath{D^0}\xspace}
\def\Dzb     {\ensuremath{\Dbar^0}\xspace}
\def\DzDzb   {\ensuremath{\Dz {\kern -0.16em \Dzb}}\xspace}
\def\Dp      {\ensuremath{D^+}\xspace}
\def\Dm      {\ensuremath{D^-}\xspace}
\def\Dpm     {\ensuremath{D^\pm}\xspace}
\def\Dmp     {\ensuremath{D^\mp}\xspace}
\def\DpDm    {\ensuremath{\Dp {\kern -0.16em \Dm}}\xspace}
\def\Dstar   {\ensuremath{D^*}\xspace}
\def\Dstarb  {\ensuremath{\Dbar^*}\xspace}
\def\Dstarz  {\ensuremath{D^{*0}}\xspace}
\def\Dstarzb {\ensuremath{\Dbar^{*0}}\xspace}
\def\Dstarp  {\ensuremath{D^{*+}}\xspace}
\def\Dstarm  {\ensuremath{D^{*-}}\xspace}
\def\Dstarpm {\ensuremath{D^{*\pm}}\xspace}
\def\Dstarmp {\ensuremath{D^{*\mp}}\xspace}
\def\DorDsz  {\ensuremath{D^{(*)0}}\xspace}
\def\DorDsp  {\ensuremath{D^{(*)+}}\xspace}
\def\Ds      {\ensuremath{D^+_s}\xspace}
\def\Dsb     {\ensuremath{\Dbar^+_s}\xspace}
\def\Dss     {\ensuremath{D^{*+}_s}\xspace}

% Obsolete
\newcommand{\dstr}{\ensuremath{\Dstar}\xspace}
\newcommand{\dstrstr}{\ensuremath{D^{**}}\xspace}
\newcommand{\dsp}{\ensuremath{\Dstarp}\xspace}
\newcommand{\dsm}{\ensuremath{\Dstarm}\xspace}
\newcommand{\dsz}{\ensuremath{\Dstarz}\xspace}


\def\B       {\ensuremath{B}\xspace}
%%% do NOT use ensuremath here
\def\Bbar    {\kern 0.18em\overline{\kern -0.18em B}{}\xspace}
\def\Bb      {\ensuremath{\Bbar}\xspace}
\def\BB      {\ensuremath{B\Bbar}\xspace} 
\def\Bz      {\ensuremath{B^0}\xspace}
\def\Bzb     {\ensuremath{\Bbar^0}\xspace}
\def\BzBzb   {\ensuremath{\Bz {\kern -0.16em \Bzb}}\xspace}
\def\Bu      {\ensuremath{B^+}\xspace}
\def\Bub     {\ensuremath{B^-}\xspace}
\def\Bp      {\ensuremath{\Bu}\xspace}
\def\Bm      {\ensuremath{\Bub}\xspace}
\def\Bpm     {\ensuremath{B^\pm}\xspace}
\def\Bmp     {\ensuremath{B^\mp}\xspace}
\def\BpBm    {\ensuremath{\Bu {\kern -0.16em \Bub}}\xspace}
\def\Bs      {\ensuremath{B_s}\xspace}
\def\Bsb     {\ensuremath{\Bbar_s}\xspace}


%%%%%%%%%%%%%%%%%%%%%%%%%%%%%%%%%%%%%%%%%%%%%%%%%%
%%%%%%%%%%%%%%%%%%%%% ONIA %%%%%%%%%%%%%%%%%%%%%%%
%%%%%%%%%%%%%%%%%%%%%%%%%%%%%%%%%%%%%%%%%%%%%%%%%%

\def\jpsi     {\ensuremath{{J\mskip -3mu/\mskip -2mu\psi\mskip 2mu}}\xspace}
\def\psitwos  {\ensuremath{\psi{(2S)}}\xspace}
\def\psiprpr  {\ensuremath{\psi(3770)}\xspace}
\def\etac     {\ensuremath{\eta_c}\xspace}
\def\chiczero {\ensuremath{\chi_{c0}}\xspace}
\def\chicone  {\ensuremath{\chi_{c1}}\xspace}
\def\chictwo  {\ensuremath{\chi_{c2}}\xspace}
\mathchardef\Upsilon="7107
\def\Y#1S{\ensuremath{\Upsilon{(#1S)}}\xspace}% no space before {...}!
\def\OneS  {\Y1S}
\def\TwoS  {\Y2S}
\def\ThreeS{\Y3S}
\def\FourS {\Y4S}
\def\FiveS {\Y5S}

%\def\chic1{\ensuremath{\chi_{c1}}}
%\def\chic2{\ensuremath{\chi_{c2}}}
%\def\chic3{\ensuremath{\chi_{c3}}}
\def\chic#1{\ensuremath{\chi_{c#1}}\xspace} % dbm


%%%%%%%%%%%%%%%%%%%%%%%%%%%%%%%%%%%%%%%%%%%%%%%%%%
%%%%%%%%%%%%%%%%%%% BARYONS %%%%%%%%%%%%%%%%%%%%%%
%%%%%%%%%%%%%%%%%%%%%%%%%%%%%%%%%%%%%%%%%%%%%%%%%%

\def\proton      {\ensuremath{p}\xspace}
\def\antiproton  {\ensuremath{\overline p}\xspace}
\def\neutron     {\ensuremath{n}\xspace}
\def\antineutron {\ensuremath{\overline n}\xspace}

\mathchardef\Deltares="7101
\mathchardef\Xi="7104
\mathchardef\Lambda="7103
\mathchardef\Sigma="7106
\mathchardef\Omega="710A

%%% do NOT use ensuremath here
\def\Deltabar{\kern 0.25em\overline{\kern -0.25em \Deltares}{}\xspace}
\def\Lbar{\kern 0.2em\overline{\kern -0.2em\Lambda\kern 0.05em}\kern-0.05em{}\xspace}
\def\Sigbar{\kern 0.2em\overline{\kern -0.2em \Sigma}{}\xspace}
\def\Xibar{\kern 0.2em\overline{\kern -0.2em \Xi}{}\xspace}
\def\Obar{\kern 0.2em\overline{\kern -0.2em \Omega}{}\xspace}
\def\Nbar{\kern 0.2em\overline{\kern -0.2em N}{}\xspace}
\def\Xb{\kern 0.2em\overline{\kern -0.2em X}{}\xspace}

\def\X {\ensuremath{X}\xspace}


%%%%%%%%%%%%%%%%%%%%%%%%%%%%%%%%%%%%%%%%%%%%%%%%%%
%%%%%%%%%% TAU BRANCHING FRACTIONS %%%%%%%%%%%%%%%
%%%%%%%%%%%%%%%%%%%%%%%%%%%%%%%%%%%%%%%%%%%%%%%%%%

\def\BR         {{\ensuremath{\cal B}\xspace}}
\def\BRtauptoe  {\ensuremath{\BR(\taup \to \ep)}\xspace}
\def\BRtaumtoe  {\ensuremath{\BR(\taum \to \en)}\xspace}
\def\BRtauptomu {\ensuremath{\BR(\taup \to \mup)}\xspace}
\def\BRtaumtomu {\ensuremath{\BR(\taum \to \mun)}\xspace}



%%%%%%%%%%%%%%%%%%%%%%%%%%%%%%%%%%%%%%%%%%%%%%%%%%
%%%%%%%%%%%  LIGHT HADRON DECAYS %%%%%%%%%%%%%%%%%
%%%%%%%%%%%%%%%%%%%%%%%%%%%%%%%%%%%%%%%%%%%%%%%%%%

\newcommand{\etaprepp}{\ensuremath{\etapr \to \eta \pipi}\xspace}
\newcommand{\etaprrg} {\ensuremath{\etapr \to \rho^0 \g}\xspace}


%%%%%%%%%%%%%%%%%%%%%%%%%%%%%%%%%%%%%%%%%%%%%%%%%%
%%%%%%%%%%%%%%%%  B DECAYS   %%%%%%%%%%%%%%%%%%%%%
%%%%%%%%%%%%%%%%%%%%%%%%%%%%%%%%%%%%%%%%%%%%%%%%%%

\def\bpsiks     {\ensuremath{\Bz \to \jpsi \KS}\xspace}
\def\bpsikst    {\ensuremath{\Bz \to \jpsi \Kstar}\xspace}
\def\bpsikl     {\ensuremath{\Bz \to \jpsi \KL}\xspace}
\def\bpsikzeropi{\ensuremath{\Bz \to \jpsi \Kstarz (\to \KL \piz)}\xspace}
\def\bpsikpluspi{\ensuremath{\Bu \to \jpsi \Kstarp (\to \KL \pip)}\xspace}
\def\bpsikpi    {\ensuremath{\Bz/\Bzb \to \jpsi (\to \mumu) \Kpm \pimp}\xspace}
\def\bupsik     {\ensuremath{\Bpm \to \jpsi (\to \mumu) \Kpm}\xspace}
\def\bpsiX      {\ensuremath{\Bz \to \jpsi \X}\xspace}

\def\Bzbtomu    {\ensuremath{\Bzb \to \mu \X}\xspace}
\def\Bzbtox     {\ensuremath{\Bzb \to \X}\xspace}
\def\Bztopipi   {\ensuremath{\Bz \to \pipi}\xspace}
\def\Bztokpi    {\ensuremath{\Bz \to \Kpm \pimp}\xspace}
\def\Bztorhopi  {\ensuremath{\Bz \to \rho^+ \pim}\xspace}
\def\Bztorhorho {\ensuremath{\Bz \to \rho \rho}\xspace}
\def\Bztokrho   {\ensuremath{\Bz \to K \rho}\xspace}
\def\Bztokstpi  {\ensuremath{\Bz \to \Kstar \pi}\xspace}
\def\Bztoapi    {\ensuremath{\Bz \to a_1 \pi}\xspace}
\def\Bztodd     {\ensuremath{\Bz \to \DpDm}\xspace}
\def\Bztodstd   {\ensuremath{\Bz \to \Dstarp \Dm}\xspace}
\def\Bztodstdst {\ensuremath{\Bz \to \Dstarp \Dstarm}\xspace}

\def\BtoDK      {\ensuremath{B \to DK}\xspace}
\def\Btodstlnu  {\ensuremath{B \to \Dstar \ell \nu}\xspace}
\def\Btodstdlnu {\ensuremath{B \to \Dstar(D) \ell \nu}\xspace}
\def\Btorholnu  {\ensuremath{B \to \rho \ell \nu}\xspace}
\def\Btopilnu   {\ensuremath{B \to \pi \ell \nu}\xspace}

\def\Btoetah    {\ensuremath{B \to \eta h}\xspace}
\def\Btoetaph   {\ensuremath{B \to \etapr h}\xspace}

\newcommand{\Betaprks}{\ensuremath{\Bz \to \etapr \KS}\xspace}
\newcommand{\Betaprkz}{\ensuremath{\Bz \to \etapr \Kz}\xspace}

\def\btosgam    {\ensuremath{b \to s \g}\xspace}
\def\btodgam    {\ensuremath{b \to d \g}\xspace}
\def\btosll     {\ensuremath{b \to s \ellell}\xspace}
\def\btosnunu   {\ensuremath{b \to s \nunub}\xspace}
\def\btosgaga   {\ensuremath{b \to s \gaga}\xspace}
\def\btosglue   {\ensuremath{b \to s g}\xspace}


%%%%%%%%%%%%%%%%%%%%%%%%%%%%%%%%%%%%%%%%%%%%%%%%%%
%%%%%%%%%%%%%%%%  Y(4S) DECAYS   %%%%%%%%%%%%%%%%%
%%%%%%%%%%%%%%%%%%%%%%%%%%%%%%%%%%%%%%%%%%%%%%%%%%

\def\upsbb   {\ensuremath{\FourS \to \BB}\xspace}
\def\upsbzbz {\ensuremath{\FourS \to \BzBzb}\xspace}
\def\upsbpbm {\ensuremath{\FourS \to \BpBm}\xspace}
\def\upspsikl{\ensuremath{\FourS \to (\bpsikl) (\Bzbtox)}\xspace}


%%%%%%%%%%%%%%%%%%%%%%%%%%%%%%%%%%%%%%%%%%%%%%%%%%
%%%%%%%%%%%%%%%%  TAU DECAYS   %%%%%%%%%%%%%%%%%%%
%%%%%%%%%%%%%%%%%%%%%%%%%%%%%%%%%%%%%%%%%%%%%%%%%%

\def\tauptoe    {\ensuremath{\taup \to \ep \nunub}\xspace}
\def\taumtoe    {\ensuremath{\taum \to \en \nunub}\xspace}
\def\tauptomu   {\ensuremath{\taup \to \mup \nunub}\xspace}
\def\taumtomu   {\ensuremath{\taum \to \mun \nunub}\xspace}
\def\tauptopi   {\ensuremath{\taup \to \pip \nub}\xspace}
\def\taumtopi   {\ensuremath{\taum \to \pim \nu}\xspace}


%%%%%%%%%%%%%%%%%%%%%%%%%%%%%%%%%%%%%%%%%%%%%%%%%%
%%%%%%%%%%%%%% GAMMA-GAMMA REACTIONS %%%%%%%%%%%%%
%%%%%%%%%%%%%%%%%%%%%%%%%%%%%%%%%%%%%%%%%%%%%%%%%%

\def\ggtopi     {\ensuremath{\gaga \to \pipi}\xspace}
\def\ggtopiz    {\ensuremath{\gaga \to \ppz}\xspace}
\def\ggstox     {\ensuremath{\ggstar \to \X(1420) \to \kaon \kaon \pi}\xspace}
\def\ggstoeta   {\ensuremath{\ggstar \to \eta(550) \to \pipi \piz}\xspace}


%%%%%%%%%%%%%%%%%%%%%%%%%%%%%%%%%%%%%%%%%%%%%%%%%%
%%%%%%%%%%%%%%%%%   KINEMATICS    %%%%%%%%%%%%%%%%
%%%%%%%%%%%%%%%%%%%%%%%%%%%%%%%%%%%%%%%%%%%%%%%%%%

\def\ptot       {\mbox{$p$}\xspace}
%\def\pxy        {\mbox{$p_{\rm t}$}
\def\pxy        {\mbox{$p_T$}\xspace}
%\def\pt         {\mbox{$p_{\rm t}$}\xspace}
\def\pt         {\mbox{$p_T$}\xspace}
\def\mes        {\mbox{$m_{\rm ES}$}\xspace}
\def\mec        {\mbox{$m_{\rm EC}$}\xspace}
\def\DeltaE     {\mbox{$\Delta E$}\xspace}

\def\pbcm {\ensuremath{p^*_{\Bz}}\xspace}


%%%%%%%%%%%%%%%%%%%%%%%%%%%%%%%%%%%%%%%%%%%%%%%%%%
%%%%%%%%%%%%%%%%%   GEOMETRY    %%%%%%%%%%%%%%%%%%
%%%%%%%%%%%%%%%%%%%%%%%%%%%%%%%%%%%%%%%%%%%%%%%%%%

\def\mphi       {\mbox{$\phi$}\xspace}
\def\mtheta     {\mbox{$\theta$}\xspace}
\def\ctheta     {\mbox{$\cos\theta$}\xspace}


%%%%%%%%%%%%%%%%%%%%%%%%%%%%%%%%%%%%%%%%%%%%%%%%%%
%%%%%%%%%%%% ENERGY AND MOMENTUM %%%%%%%%%%%%%%%%%
%%%%%%%%%%%%%%%%%%%%%%%%%%%%%%%%%%%%%%%%%%%%%%%%%%

\newcommand{\tev}{\ensuremath{\mathrm{\,Te\kern -0.1em V}}\xspace}
\newcommand{\gev}{\ensuremath{\mathrm{\,Ge\kern -0.1em V}}\xspace}
\newcommand{\mev}{\ensuremath{\mathrm{\,Me\kern -0.1em V}}\xspace}
\newcommand{\kev}{\ensuremath{\mathrm{\,ke\kern -0.1em V}}\xspace}
\newcommand{\ev}{\ensuremath{\mathrm{\,e\kern -0.1em V}}\xspace}
\newcommand{\gevc}{\ensuremath{{\mathrm{\,Ge\kern -0.1em V\!/}c}}\xspace}
\newcommand{\mevc}{\ensuremath{{\mathrm{\,Me\kern -0.1em V\!/}c}}\xspace}
\newcommand{\gevcc}{\ensuremath{{\mathrm{\,Ge\kern -0.1em V\!/}c^2}}\xspace}
\newcommand{\mevcc}{\ensuremath{{\mathrm{\,Me\kern -0.1em V\!/}c^2}}\xspace}
%\def\ev   {\ensuremath{\rm \,e\kern -0.08em V}}
%\def\kev  {\ensuremath{\rm \,ke\kern -0.08em V}} 
%\def\mev  {\ensuremath{\rm \,Me\kern -0.08em V}} 
%\def\gev  {\ensuremath{\rm \,Ge\kern -0.08em V}} 
%\def\gevc {\ensuremath{\rm \,Ge\kern -0.08em V\!/c}} 
%\def\gevc {\ensuremath{{\rm \,Ge\kern -0.08em V\!/}c}} 
%\def\tev  {\ensuremath{\rm \,Te\kern -0.08em V}}
%\def\mevc {\ensuremath{\rm \,Me\kern -0.08em V\!/c}} 
%\def\mevc {\ensuremath{{\rm \,Me\kern -0.08em V\!/}c}} 
%\def\gevcc{\ensuremath{\rm \,Ge\kern -0.08em V\!/c^2}} 
%\def\mevcc{\ensuremath{\rm \,Me\kern -0.08em V\!/c^2}} 
%\def\gevcc{\ensuremath{{\rm \,Ge\kern -0.08em V\!/}c^2}} 
%\def\mevcc{\ensuremath{{\rm \,Me\kern -0.08em V\!/}c^2}} 


%%%%%%%%%%%%%%%%%%%%%%%%%%%%%%%%%%%%%%%%%%%%%%%%%%
%%%%%%%%%%%% DISTANCE AND AREA %%%%%%%%%%%%%%%%%%%
%%%%%%%%%%%%%%%%%%%%%%%%%%%%%%%%%%%%%%%%%%%%%%%%%%

\def\syin {\ensuremath{^{\prime\prime}}\xspace}
\def\inch   {\ensuremath{\rm \,in}\xspace} % \in is taken
\def\ft   {\ensuremath{\rm \,ft}\xspace}
\def\km   {\ensuremath{\rm \,km}\xspace}
\def\m    {\ensuremath{\rm \,m}\xspace}
\def\cm   {\ensuremath{\rm \,cm}\xspace}
\def\cma  {\ensuremath{{\rm \,cm}^2}\xspace}
\def\mm   {\ensuremath{\rm \,mm}\xspace}
\def\mma  {\ensuremath{{\rm \,mm}^2}\xspace}
%\def\mum  {\ensuremath{\rm \,\mum}\xspace}
\def\mum  {\ensuremath{\,\mu\rm m}\xspace}%% mu meter 
%\def\muma {\ensuremath{\rm \,\mum}^2\xspace}
\def\muma       {\ensuremath{\,\mu\rm m^2}\xspace}
\def\nm   {\ensuremath{\rm \,nm}\xspace}
\def\fm   {\ensuremath{\rm \,fm}\xspace}
\def\nm         {\ensuremath{\rm \,nm}\xspace}   %% nanometer
\def\nb         {\ensuremath{\rm \,nb}\xspace}
%
\def\barn{\ensuremath{\rm \,b}\xspace}
\def\barnhyph{\ensuremath{\rm -b}\xspace}
\def\mbarn{\ensuremath{\rm \,mb}\xspace}
\def\mbarnhyph{\ensuremath{\rm -mb}\xspace}
\def\pb {\ensuremath{\rm \,pb}\xspace}
\def\invpb {\ensuremath{\mbox{\,pb}^{-1}}\xspace}
\def\fb   {\ensuremath{\mbox{\,fb}}\xspace}
\def\invfb   {\ensuremath{\mbox{\,fb}^{-1}}\xspace}


%%%%%%%%%%%%%%%%%%%%%%%%%%%%%%%%%%%%%%%%%%%%%%%%%%
%%%%%%%%%%%% TIME AND MASS  %%%%%%%%%%%%%%%%%%%%%%
%%%%%%%%%%%%%%%%%%%%%%%%%%%%%%%%%%%%%%%%%%%%%%%%%%

\def\mus  {\ensuremath{\rm \,\mus}\xspace}
\def\ns   {\ensuremath{\rm \,ns}\xspace}
\def\ps   {\ensuremath{\rm \,ps}\xspace}
\def\fs   {\ensuremath{\rm \,fs}\xspace}
\def\gm   {\ensuremath{\rm \,g}\xspace}

%%\def\s{\ensuremath{\rm {\,s}}} %% second - this displays nothing  - why?
\def\sec{\ensuremath{\rm {\,s}}\xspace}       %% second - this works - jw 4/19
\def\ms         {\ensuremath{{\rm \,ms}}\xspace}     %% millisecond
\def\mus        {\ensuremath{\,\mu{\rm s}}\xspace}    %% microsecond
\def\ns         {\ensuremath{{\rm \,ns}}\xspace}      %% nanosecond
\def\ps         {\ensuremath{{\rm \,ps}}\xspace}  %% picosecond


%%%%%%%%%%%%%%%%%%%%%%%%%%%%%%%%%%%%%%%%%%%%%%%%%%
%%%%%%%%%%%%   MISCELLANEOUS %%%%%%%%%%%%%%%%%%%%%
%%%%%%%%%%%%%%%%%%%%%%%%%%%%%%%%%%%%%%%%%%%%%%%%%%

\def\Xrad {\ensuremath{X_0}\xspace}
\def\NIL{\ensuremath{\lambda_{int}}\xspace}
\let\dgr\degrees


%\def\m          {\ensuremath{\rm \,m}}    %% meter
%\def\ma         {\ensuremath{\rm \,m}^2}  %% meter squared
%\def\cm         {\ensuremath{\rm \,cm}}   %% centimeter
%\def\cma        {\ensuremath{\rm \,cm}^2} %% centimeter squared
\def\cms         {\ensuremath{{\rm \,cm}^{-2} {\rm s}^{-1}}\xspace}
%\def\mm         {\ensuremath{\rm \,mm}}   %% millimeter
%\def\mma        {\ensuremath{\rm \,mm}^2} %% millimeter squared
%\def\mum        {\ensuremath{\,\mu\rm m}} %% mu meter
%\def\muma       {\ensuremath{\,\mu\rm m^2}}


\def\mic  {\ensuremath{\,\mu{\rm C}}\xspace}
\def\krad {\ensuremath{\rm \,krad}\xspace}
\def\cmc  {\ensuremath{{\rm \,cm}^3}\xspace}
\def\yr   {\ensuremath{\rm \,yr}\xspace}
\def\hr   {\ensuremath{\rm \,hr}\xspace}
\def\degc {\ensuremath{^\circ}{C}\xspace}
\def\degk {\ensuremath {\rm K}\xspace}
\def\degrees{\ensuremath{^{\circ}}\xspace}
\def\mrad{\ensuremath{\rm \,mrad}\xspace}               %% milliradian
\def\rad{\ensuremath{\rm \,rad}\xspace}
\def\mradhyph{\ensuremath{\rm -mr}\xspace}
%
\def\sx    {\ensuremath{\sigma_x}\xspace}    
\def\sy    {\ensuremath{\sigma_y}\xspace}   
\def\sz    {\ensuremath{\sigma_z}\xspace}    

%\renewcommand{\bar}[1]{\overline{#1}}  

% Some more (from Helen)
%\def\O{{\ensuremath{\cal O}}}  !!! This is a predefined LaTeX symbol !!!
\def\order{{\ensuremath{\cal O}}\xspace}
\def\L{{\ensuremath{\cal L}}\xspace}
\def\calL{{\ensuremath{\cal L}}\xspace}
%\def\S{{\ensuremath{\cal S}}}  !!! This is a predefined LaTeX symbol !!!
\def\calS{{\ensuremath{\cal S}}\xspace}
\def\calA{{\ensuremath{\cal A}}\xspace}
\def\calD{{\ensuremath{\cal D}}\xspace}
\def\calR{{\ensuremath{\cal R}}\xspace}

%% Arrows:
\def\ra                 {\ensuremath{\rightarrow}\xspace}
\def\to                 {\ensuremath{\rightarrow}\xspace}


\newcommand{\stat}{\ensuremath{\mathrm{(stat)}}\xspace}
\newcommand{\syst}{\ensuremath{\mathrm{(syst)}}\xspace}


\def\pep2{PEP-II}
\def\BF{$B$ Factory}
\def\abf {asymmetric \BF}
\def\sx    {\ensuremath{\sigma_x}\xspace}     
\def\sy    {\ensuremath{\sigma_y}\xspace}   
\def\sz    {\ensuremath{\sigma_z}\xspace}


\newcommand{\inverse}{\ensuremath{^{-1}}\xspace}
\newcommand{\dedx}{\ensuremath{\mathrm{d}\hspace{-0.1em}E/\mathrm{d}x}\xspace}
\newcommand{\chisq}{\ensuremath{\chi^2}\xspace}
\newcommand{\delm}{\ensuremath{m_{\dstr}-m_{\dz}}\xspace}
\newcommand{\lum} {\ensuremath{\mathcal{L}}\xspace}

\def\gsim{{~\raise.15em\hbox{$>$}\kern-.85em
          \lower.35em\hbox{$\sim$}~}\xspace}
\def\lsim{{~\raise.15em\hbox{$<$}\kern-.85em
          \lower.35em\hbox{$\sim$}~}\xspace}

\def\qsq                {\ensuremath{q^2}\xspace}


% Data processing
\def\kbytes     {\ensuremath{{\rm \,kbytes}}\xspace}
\def\kbsps      {\ensuremath{{\rm \,kbytes/s}}\xspace}
\def\kbits      {\ensuremath{{\rm \,kbits}}\xspace}
\def\kbsps      {\ensuremath{{\rm \,kbits/s}}\xspace}
\def\mbsps      {\ensuremath{{\rm \,Mbits/s}}\xspace}
\def\mbytes     {\ensuremath{{\rm \,Mbytes}}\xspace}
\def\mbps       {\ensuremath{{\rm \,Mbyte/s}}\xspace}
\def\mbsps      {\ensuremath{{\rm \,Mbytes/s}}\xspace}
\def\gbsps      {\ensuremath{{\rm \,Gbits/s}}\xspace}
\def\gbytes     {\ensuremath{{\rm \,Gbytes}}\xspace}
\def\gbsps      {\ensuremath{{\rm \,Gbytes/s}}\xspace}
\def\tbytes     {\ensuremath{{\rm \,Tbytes}}\xspace}
\def\tbpy       {\ensuremath{{\rm \,Tbytes/yr}}\xspace}
%




% QCD parameters

\newcommand{\as}{\ensuremath{\alpha_{\scriptscriptstyle S}}\xspace}
\newcommand{\MSb}{\ensuremath{\overline{\mathrm{MS}}}\xspace}
\newcommand{\LMSb}{%
  \ensuremath{\Lambda_{\overline{\scriptscriptstyle\mathrm{MS}}}}\xspace
}

% Electroweak parameters

\newcommand{\tw}{\ensuremath{\theta_{\scriptscriptstyle W}}\xspace}
\newcommand{\twb}{%
  \ensuremath{\overline{\theta}_{\scriptscriptstyle W}}\xspace
}
\newcommand{\Afb}[1]{{\ensuremath{A_{\scriptscriptstyle FB}^{#1}}}\xspace}
\newcommand{\gv}[1]{{\ensuremath{g_{\scriptscriptstyle V}^{#1}}}\xspace}
\newcommand{\ga}[1]{{\ensuremath{g_{\scriptscriptstyle A}^{#1}}}\xspace}
\newcommand{\gvb}[1]{{\ensuremath{\overline{g}_{\scriptscriptstyle V}^{#1}}}\xspace}
\newcommand{\gab}[1]{{\ensuremath{\overline{g}_{\scriptscriptstyle A}^{#1}}}\xspace}


% CKM, CP violation

\def\eps{\varepsilon\xspace}
\def\epsK{\varepsilon_K\xspace}
\def\epsB{\varepsilon_B\xspace}
\def\epsp{\varepsilon^\prime_K\xspace}

\def\CP                {\ensuremath{C\!P}\xspace}
%\def\CPT               {\ensuremath{C\!P\!T}\xspace}
\def\CPT               {\ensuremath{C\!PT}\xspace} % Looks better without \!

\def\epstag  {\ensuremath{\varepsilon_{\rm tag}}\xspace}
\def\tagfac  {\ensuremath{\epstag(1-2w)^2}\xspace}

\def\rhobar {\ensuremath{\overline \rho}\xspace}
\def\etabar {\ensuremath{\overline \eta}\xspace}
%\def\paramest {\ensuremath{{\hat A}, {\hat \rho}, {\hat \eta} }}
%\def\ssparamest {\ensuremath{{\hat A}, {\hat {\sin 2 \alpha}}, 
%{\hat {\sin 2 \beta}} }}
\def\meas {\ensuremath{|V_{cb}|, |\frac{V_{ub}}{V_{cb}}|, 
|\varepsilon_K|, \Delta m_{B_d}}\xspace}

\def\Vud  {\ensuremath{|V_{ud}|}\xspace}
\def\Vcd  {\ensuremath{|V_{cd}|}\xspace}
\def\Vtd  {\ensuremath{|V_{td}|}\xspace}
\def\Vus  {\ensuremath{|V_{us}|}\xspace}
\def\Vcs  {\ensuremath{|V_{cs}|}\xspace}
\def\Vts  {\ensuremath{|V_{ts}|}\xspace}
\def\Vub  {\ensuremath{|V_{ub}|}\xspace}
\def\Vcb  {\ensuremath{|V_{cb}|}\xspace}
\def\Vtb  {\ensuremath{|V_{tb}|}\xspace}

%\def\sa{${\sin\! 2 \alpha  }$\xspace}
%\def\sb{${\sin\! 2 \beta   }$\xspace}
%\def\sg{${\sin\! 2 \gamma  }$\xspace}

% added by Gautier for tagging, tagmix, and sin2beta
\def\stwoa{\ensuremath{\sin\! 2 \alpha  }\xspace}
\def\stwob{\ensuremath{\sin\! 2 \beta   }\xspace}
\def\stwog{\ensuremath{\sin\! 2 \gamma  }\xspace}
\def\mistag{\ensuremath{w}\xspace}
\def\dilution{\ensuremath{\cal D}\xspace}
\def\deltaz{\ensuremath{{\rm \Delta}z}\xspace}
\def\deltat{\ensuremath{{\rm \Delta}t}\xspace}
\def\deltamd{\ensuremath{{\rm \Delta}m_d}\xspace}

\newcommand{\fsubd}{\ensuremath{f_D}}\xspace
\newcommand{\fds}{\ensuremath{f_{D_s}}\xspace}
\newcommand{\fsubb}{\ensuremath{f_B}\xspace}
\newcommand{\fbd}{\ensuremath{f_{B_d}}\xspace}
\newcommand{\fbs}{\ensuremath{f_{B_s}}\xspace}
\newcommand{\bsubb}{\ensuremath{B_B}\xspace}
\newcommand{\bbd}{\ensuremath{B_{B_d}}\xspace}
\newcommand{\bbs}{\ensuremath{B_{B_s}}\xspace}
\newcommand{\rgbb}{\ensuremath{\hat{B}_B}\xspace}
\newcommand{\rgbbd}{\ensuremath{\hat{B}_{B_d}}\xspace}
\newcommand{\rgbbs}{\ensuremath{\hat{B}_{B_s}}\xspace}
\newcommand{\rgbk}{\ensuremath{\hat{B}_K}\xspace}
\newcommand{\lqcd}{\ensuremath{\Lambda_{\mathrm{QCD}}}\xspace}

%\newcommand{\eqref}[1]{Eq.~(\ref{eq:#1})}
\newcommand{\secref}[1]{Section~\ref{sec:#1}}
\newcommand{\subsecref}[1]{Section~\ref{subsec:#1}}
\newcommand{\figref}[1]{Figure~\ref{fig:#1}}
\newcommand{\tabref}[1]{Table~\ref{tab:#1}}






% Journal References

% These bases are useful for ``submitted to'' when no volume is needed
\newcommand{\epjBase}        {Eur.\ Phys.\ Jour.\xspace}
\newcommand{\jprlBase}       {Phys.\ Rev.\ Lett.\xspace}
\newcommand{\jprBase}        {Phys.\ Rev.\xspace}
\newcommand{\jplBase}        {Phys.\ Lett.\xspace}
\newcommand{\nimBaseA}       {Nucl.\ Instr.\ Meth.\xspace}
\newcommand{\nimBaseB}       {Nucl.\ Instr.\ and Meth.\xspace}
\newcommand{\nimBaseC}       {Nucl.\ Instr.\ and Methods\xspace}
\newcommand{\nimBaseD}       {Nucl.\ Instrum.\ Methods\xspace}
\newcommand{\npBase}         {Nucl.\ Phys.\xspace}
\newcommand{\zpBase}         {Z.\ Phys.\xspace}


\newcommand{\apas}      [1]  {{Acta Phys.\ Austr.\ Suppl.\ {\bf #1}}}
\newcommand{\app}       [1]  {{Acta Phys.\ Polon.\ {\bf #1}}}
\newcommand{\ace}       [1]  {{Adv.\ Cry.\ Eng.\ {\bf #1}}}
\newcommand{\anp}       [1]  {{Adv.\ Nucl.\ Phys.\ {\bf #1}}}
\newcommand{\annp}      [1]  {{Ann.\ Phys.\ {\bf #1}}}
\newcommand{\araa}      [1]  {{Ann.\ Rev.\ Astr.\ Ap.\ {\bf #1}}}
\newcommand{\arnps}     [1]  {{Ann.\ Rev.\ Nucl.\ Part.\ Sci.\ {\bf #1}}}
\newcommand{\arns}      [1]  {{Ann.\ Rev.\ Nucl.\ Sci.\ {\bf #1}}}
\newcommand{\appopt}    [1]  {{Appl.\ Opt.\ {\bf #1}}}
\newcommand{\japj}      [1]  {{Astro.\ Phys.\ J.\ {\bf #1}}}
\newcommand{\baps}      [1]  {{Bull.\ Am.\ Phys.\ Soc.\ {\bf #1}}}
\newcommand{\seis}      [1]  {{Bull.\ Seismological Soc.\ of Am.\ {\bf #1}}}
\newcommand{\cmp}       [1]  {{Commun.\ Math.\ Phys.\ {\bf #1}}}
\newcommand{\cnpp}      [1]  {{Comm.\ Nucl.\ Part.\ Phys.\ {\bf #1}}}
\newcommand{\cpc}       [1]  {{Comput.\ Phys.\ Commun.\ {\bf #1}}}
\newcommand{\epj}       [1]  {\epjBase\ {\bf #1}}
%\newcommand{\epjc}      [1]  {{Eur.\ Phys.\ Jour.\ C~{\bf #1}}}
\newcommand{\epjc}      [1]  {\epjBase\ C~{\bf #1}}
\newcommand{\fizika}    [1]  {{Fizika~{\bf #1}}}
\newcommand{\fp}        [1]  {{Fortschr.\ Phys.\ {\bf #1}}}
\newcommand{\ited}      [1]  {{IEEE Trans.\ Electron.\ Devices~{\bf #1}}}
\newcommand{\itns}      [1]  {{IEEE Trans.\ Nucl.\ Sci.\ {\bf #1}}}
\newcommand{\ijqe}      [1]  {{IEEE J.\ Quantum Electron.\ {\bf #1}}}
\newcommand{\ijmp}      [1]  {{Int.\ Jour.\ Mod.\ Phys.\ {\bf #1}}}
\newcommand{\ijmpa}     [1]  {{Int.\ J.\ Mod.\ Phys.\ {\bf A{\bf #1}}}}
\newcommand{\jl}        [1]  {{JETP Lett.\ {\bf #1}}}
\newcommand{\jetp}      [1]  {{JETP~{\bf #1}}}
\newcommand{\jpg}       [1]  {{J.\ Phys.\ {\bf G{\bf #1}}}}
\newcommand{\jap}       [1]  {{J.\ Appl.\ Phys.\ {\bf #1}}}
\newcommand{\jmp}       [1]  {{J.\ Math.\ Phys.\ {\bf #1}}}
\newcommand{\jmes}      [1]  {{J.\ Micro.\ Elec.\ Sys.\ {\bf #1}}}
%\newcommand{\josa}      [1]  {{J.\ Opt.\ Soc.\ Am.\ {\bf #1}}}
\newcommand{\mpl}       [1]  {{Mod.\ Phys.\ Lett.\ {\bf #1}}}

%\newcommand{\nim}       [1]  {{Nucl.\ Instr.\ and Methods~{\bf #1}}}
\newcommand{\nim}       [1]  {\nimBaseC~{\bf #1}}
%\newcommand{\nima}      [1]  {{Nucl.\ Instr.\ Methods~{\bf A{\bf #1}}}}
\newcommand{\nima}      [1]  {\nimBaseC~A~{\bf #1}}

%\newcommand{\np}        [1]  {{Nucl.\ Phys.\ {\bf #1}}}
\newcommand{\np}        [1]  {\npBase\ {\bf #1}}
%\newcommand{\npb}       [1]  {{Nucl.\ Phys.\ {\bf B{\bf #1}}}}
\newcommand{\npb}       [1]  {\npBase\ B~{\bf #1}}
\newcommand{\npps}      [1]  {{Nucl.\ Phys.\ Proc.\ Suppl.\ {\bf #1}}}
\newcommand{\npaps}     [1]  {{Nucl.\ Phys.\ A~Proc.\ Suppl.\ {\bf #1}}}
\newcommand{\npbps}     [1]  {{Nucl.\ Phys.\ B~Proc.\ Suppl.\ {\bf #1}}}

\newcommand{\ncim}      [1]  {{Nuo.\ Cim.\ {\bf #1}}}
\newcommand{\optl}      [1]  {{Opt.\ Lett.\ {\bf #1}}}
\newcommand{\optcom}    [1]  {{Opt.\ Commun.\ {\bf #1}}}
\newcommand{\partacc}   [1]  {{Particle Acclerators~{\bf #1}}}
\newcommand{\pan}       [1]  {{Phys.\ Atom.\ Nuclei~{\bf #1}}}
\newcommand{\pflu}      [1]  {{Physics of Fluids~{\bf #1}}}
\newcommand{\ptoday}    [1]  {{Physics Today~{\bf #1}}}

%\newcommand{\jpl}       [1]  {{Phys.\ Lett.\ {\bf #1}}}      % dbm
\newcommand{\jpl}       [1]  {\jplBase\ {\bf #1}}
%\newcommand{\plb}       [1]  {{Phys.\ Lett.\ B~{\bf #1}}}   % dbm
\newcommand{\plb}       [1]  {\jplBase\ B~{\bf #1}}
\newcommand{\prep}      [1]  {{Phys.\ Rep.\ {\bf #1}}}
%\newcommand{\jprl}      [1]  {{Phys.\ Rev.\ Lett.\ {\bf #1}}} % dbm
\newcommand{\jprl}      [1]  {\jprlBase\ {\bf #1}}
%\newcommand{\pr}        [1]  {{Phys.\ Rev.\ {\bf #1}}}
\newcommand{\pr}        [1]  {\jprBase\ {\bf #1}}
%\newcommand{\jpra}      [1]  {{Phys.\ Rev.\ A~{\bf #1}}}
\newcommand{\jpra}      [1]  {\jprBase\ A~{\bf #1}}
%\newcommand{\jprd}      [1]  {{Phys.\ Rev.\ D~{\bf #1}}}
\newcommand{\jprd}      [1]  {\jprBase\ D~{\bf #1}}
%\newcommand{\jpre}      [1]  {{Phys.\ Rev.\ E~{\bf #1}}}
\newcommand{\jpre}      [1]  {\jprBase\ E~{\bf #1}}

\newcommand{\prsl}      [1]  {{Proc.\ Roy.\ Soc.\ Lond.\ {\bf #1}}}
\newcommand{\ppnp}      [1]  {{Prog.\ Part.\ Nucl.\ Phys.\ {\bf #1}}}
\newcommand{\progtp}    [1]  {{Prog.\ Th.\ Phys.\ {\bf #1}}}
\newcommand{\rpp}       [1]  {{Rep.\ Prog.\ Phys.\ {\bf #1}}}
\newcommand{\jrmp}      [1]  {{Rev.\ Mod.\ Phys.\ {\bf #1}}}  % dbm
\newcommand{\rsi}       [1]  {{Rev.\ Sci.\ Instr.\ {\bf #1}}}
\newcommand{\sci}       [1]  {{Science~{\bf #1}}}
\newcommand{\sjnp}      [1]  {{Sov.\ J.\ Nucl.\ Phys.\ {\bf #1}}}
\newcommand{\spd}       [1]  {{Sov.\ Phys.\ Dokl.\ {\bf #1}}}
\newcommand{\spu}       [1]  {{Sov.\ Phys.\ Usp.\ {\bf #1}}}
\newcommand{\tmf}       [1]  {{Teor.\ Mat.\ Fiz.\ {\bf #1}}}
\newcommand{\yf}        [1]  {{Yad.\ Fiz.\ {\bf #1}}}
%\newcommand{\zp}        [1]  {{Z.\ Phys.\ {\bf #1}}}
\newcommand{\zp}        [1]  {\zpBase\ {\bf #1}}
%\newcommand{\zpc}       [1]  {{Z.\ Phys.\ C~{\bf #1}}}
\newcommand{\zpc}       [1]  {\zpBase\ C~{\bf #1}}
\newcommand{\zpr}       [1]  {{ZhETF Pis.\ Red.\ {\bf #1}}}

\newcommand{\hepex}     [1]  {hep-ex/{#1}}
\newcommand{\hepph}     [1]  {hep-ph/{#1}}
\newcommand{\hepth}     [1]  {hep-th/{#1}}


%%%%%%%%%%%%%%%%%%%% SOFTWARE PACKAGES %%%%%%%%%%%%%%%%%%%%%%%%%%%%%%%%%%%%%%%
\def\aslund     {\mbox{\tt Aslund}\xspace}
\def\bbsim      {\mbox{\tt BBsim}\xspace}
\def\beast      {\mbox{\tt Beast}\xspace}
\def\beget      {\mbox{\tt Beget}\xspace}
\def\Bta        {\mbox{\tt Beta}\xspace}
\def\betakfit   {\mbox{\tt BetaKfit}\xspace}
\def\cornelius  {\mbox{\tt Cornelius}\xspace}
\def\evtgen     {\mbox{\tt EvtGen}\xspace}
\def\euclid     {\mbox{\tt Euclid}\xspace}
\def\fitver     {\mbox{\tt FitVer}\xspace}
\def\fluka      {\mbox{\tt Fluka}\xspace}
\def\fortran    {\mbox{\tt Fortran}\xspace}
\def\gcalor     {\mbox{\tt GCalor}\xspace}
\def\geant      {\mbox{\tt GEANT}\xspace}
\def\gheisha    {\mbox{\tt Gheisha}\xspace}
\def\hemicosm   {\mbox{\tt HemiCosm}\xspace}
\def\hepevt     {\mbox{\tt{/HepEvt/}}\xspace}
\def\jetset74   {\mbox{\tt Jetset \hspace{-0.5em}7.\hspace{-0.2em}4}\xspace}
%\def\jetset     {\mbox{\tt Jetset \hspace{-0.5em}7.\hspace{-0.2em}4}}
\def\koralb     {\mbox{\tt KoralB}\xspace}
\def\minuit     {\mbox{\tt Minuit}\xspace}
\def\objegs     {\mbox{\tt Objegs}\xspace}
\def\paw        {\mbox{\tt Paw}\xspace}
\def\Root       {\mbox{\tt Root}\xspace}
\def\squaw      {\mbox{\tt Squaw}\xspace}
\def\stdhep     {\mbox{\tt StdHep}\xspace}
\def\trackerr   {\mbox{\tt TrackErr}\xspace}
\def\turtle     {\mbox{\tt Decay~Turtle}\xspace}


\newcommand{\Red} [1]  {\textcolor{red}{#1}}

\begin{document} 

\section{Semileptonic decays in EvtGen}
\label{semileptonic}
\index{semileptonic decays}

\label{Semileptonic decays}
\subsection{Introduction}
We summarize the treatment of semileptonic decays in
EvtGen.  A framework is setup such that the implementation of
a model for semileptonic form factors requires only the 
coding of the form factors themselves.  Currently, we consider
semileptonic decays from pseudoscalar mesons into 
pseudoscalar, vector, or tensor meson daughters.
In section~\ref{sec:conventions} we define the
conventions used in the implementation of semileptonic
decays.  Section~\ref{sec:framework} describes the framework
from which semileptonic form factor models can be built.  An
example of such a model is given in section~\ref{sec:examples}.
Finally, we list the form factor models implemented in EvtGen
in section~\ref{sec:models}.

\subsection{Semileptonic conventions}
\label{sec:conventions}
\index{Semileptonic decay conventions}
We have tried to follow standard descriptions of semileptonic
form factors in EvtGen.  
For decays of 
pseudoscalar mesons to pseudoscalar daughters, we describe the
form factors as
\begin{eqnarray}
<S(k) \vert V_{\mu} \vert P(p) > & = &\biggl \{ (p+k)_{\mu} - 
{{m_P^2 - m_S^2} \over {q^2}} q_{\mu} \biggr \} f_+(q^2) \\
&+& 
\biggl \{  {{m_P^2 - m_S^2} \over {q^2}} q_{\mu} \biggr \} f_0(q^2) 
\nonumber
\end{eqnarray}
$P$ denotes the parent (pseudoscalar) meson ($B$, $D$ or $B_s$) and 
$S$ denotes a pseudoscalar daughter ($D$, $K$, or $\pi$).  In this
form, we have the constraint $f_+(0) = f_0(0)$. 

\noindent
For decays to vector mesons, we use the form
\begin{eqnarray}
<V(k) \vert (V-A)_{\mu} \vert P(p) > & = & 
 \epsilon^*_{\mu} (m_P+m_V) {\rm A}_1(q^2)
- (p+k)_{\mu} ( \epsilon^* p) { { {\rm A}_2(q^2)} \over {(m_P + m_V)}} \\
&-& q_{\mu} ( \epsilon^* p) { {2 m_V} \over {q^2}} \bigl ({\rm A}_3(q^2)
- i{\rm A}_0(q^2) \bigr ) \nonumber
\\
&+&\epsilon_{\mu \nu \rho \sigma} \epsilon^{* \nu} p^{\rho} k^{\sigma}
{ {2 {\rm V}(q^2)} \over { m_P + m_V} } \nonumber
\end{eqnarray}
Where $V$ denotes a vector daughter meson ($D^*$, $K^*$ or $\rho$)
and ${\rm A}_3$ is defined as 
\begin{equation}
{\rm A}_3(q^2) = {{m_P+ m_V} \over {2 m_V}} {\rm A}_1(q^2) -
{{m_P- m_V} \over {2 m_V}} {\rm A}_2(q^2) 
\end{equation}
where ${\rm A}_0 (0) = {\rm A}_3(0)$. \nonumber


\noindent For semileptonic decays to tensor mesons ($D_2^*$), to
be denoted as $T$, we use the form
\begin{eqnarray}
<T(k) \vert (V-A)_{\mu} \vert P(p) > &=& ih(q^2) \epsilon_{\mu \nu \lambda \rho}
\epsilon^{* \nu \alpha} p_{\alpha} (p+k)^{\lambda} (p-k)^{\rho} -
k(q^2) \epsilon^{*}_{\mu \nu} p^{\nu} \\
& + &b_+(q^2) \epsilon^*_{\alpha \beta} p^{\alpha} p^{\beta} (p+k)_{\mu} -
b_-(q^2) \epsilon^*_{\alpha \beta} p^{\alpha} p^{\beta} (p-k)_{\mu} 
\nonumber
\end{eqnarray}

\subsection{Semileptonic framework}
\label{sec:framework}
There are two abstract classes responsible for the semileptonic decay
framework.  The first of these is 
{\tt EvtSemiLeptonicFF}, which 
provides the framework for the calculations of form factors.  There
are three member functions:
\index{{\tt EvtSemiLeptonicFF}}
\begin{itemize}
\item \begin{verbatim} virtual void getscalarff( int parent, int daught,
                       double t, double mass, double *fpf,
                       double *fmf ) 
\end{verbatim}
\item \begin{verbatim}  virtual void getvectorff( int parent, int daught,
                       double t, double mass, double *a1f,
                       double *a2f, double *vf, double *a0f ) 
\end{verbatim}
\item \begin{verbatim}  virtual void gettensorff( int parent, int daught,
                       double t, double mass, double *hf,
                       double *kf, double *bpf, double *bmf ) 
\end{verbatim}
\end{itemize}
\noindent All form factors models should include a class 
derived from {\tt EvtSemiLeptonicFF},  which
will implement the relevant member functions of
{\tt EvtSemiLeptonicFF}. 

\noindent These form factor implementations are used by classes
derived from the abstract class {\tt EvtSemiLeptonicAmp},
which contains the member function:
\index{{\it EvtSemiLeptonicAmp}}
\begin{itemize}
\item \begin{verbatim}   virtual void CalcAmp( EvtParticle *parent,
            EvtSemiLeptonicFF *FormFactors  ) 
\end{verbatim}
\end{itemize}
Classes to calculate the amplitude for semileptonic pseudoscalar to
pseudoscalar, pseudoscalar to vector, and 
pseudoscalar to tensor decays are 
{\tt EvtSemiLeptonicScalarAmp},
{\tt EvtSemiLeptonicVectorAmp}, and
{\tt EvtSemiLeptonicTensorAmp}, 
respectively.  These routines are passed form factors which
must follow the conventions described in section~\ref{sec:conventions}.
An example of how to use this framework is presented in 
section~\ref{sec:examples}.
 
\subsection{Examples}
\label{sec:examples}
\index{EvtHQET}
Above we have defined the conventions followed in the standard
semileptonic framework of EvtGen.  Here we show how new
form factor models can be easily implemented.  We take as an
example a heavy quark effictive theory model of 
$B \rightarrow D^* \ell \nu$ form factors, where $\ell$ is
either an electron or muon.  In this decay we can write the
form factors as:
\begin{equation}
A_1 = {{h_{A_1}(w)} \over {R^*}} \biggl ( 1 - { {q^2} \over {(M_B + M_{D^*})^2}}
\biggr )
\end{equation}
\begin{equation}
V = R_1 {{h_{A_1}(w)} \over {R^*}}
\end{equation}
\begin{equation}
A_2 = R_2 {{h_{A_1}(w)} \over {R^*}}
\end{equation}
\noindent where
\begin{equation}
h_{A_1}(w) = h_{A_1}(1) \biggl (1 - \rho_{A_1}^2(w-1) \biggr ),
\end{equation}
\begin{equation}
R^* = { {2.0 (M_B M_{D^*})^{1/2}} \over {(M_B+M_{D^*})}},
\end{equation} 
\noindent and
\begin{equation}
w = { {M_B^2 + M_{D^*}^2 -q^2} \over { 2 M_B M_{D^*}} }
\end{equation}
\noindent To implement a decay model using these form factors
we must write two classes
\begin{itemize} 
\item
\begin{verbatim}
class EvtHQETFF : public EvtSemiLeptonicFF {

public:
  EvtHQETFF(double hqetrho2, double hqetr1, double hqetr2);

  void getvectorff( int parent, int daught,
                       double t, double mass, double *a1f,
                       double *a2f, double *vf, double *a0f );

private:
  double r1;
  double rho2;
  double r2;
};
\end{verbatim}
\item
\begin{verbatim}
class EvtHQET:public  EvtDecayTBase<EvtHQET>  {

public:

  static void getName(char *name,int &narg);

  void Decay(EvtParticle *p);
  void InitProbMax();
  void Init();

private:
  EvtSemiLeptonicFF *hqetffmodel;
  EvtSemiLeptonicAmp *calcamp;
};
\end{verbatim}
\end{itemize}
\noindent This class is to implement $B \rightarrow D^* \ell \nu$
decays, so only the vector form factor routine must be
provided.  The constuctor allows R1, R2 and $\rho^2$ to
be passed in as a run time arguement.  We will discuss this
point more below.  The implementation of {\tt EvtHQETFF} is
quite simple:
\begin{verbatim}
EvtHQETFF::EvtHQETFF(double hqetrho2, double hqetr1, double hqetr2) {

  rho2 = hqetrho2;
  r1 = hqetr1;
  r2 = hqetr2;
  return;
}

void EvtHQETFF::getvectorff( int parent, int daught,
                       double t, double mass, double *a1f,
                             double *a2f, double *vf, double *a0f ){


  double mb=EvtPDL::nom_mass(parent);
  double w = ((mb*mb)+(mass*mass)-t)/(2.0*mb*mass);

// Form factors have a general form, with parameters passed in
// from the arguements.

  double rstar = ( 2.0*sqrt(mb*mass))/(mb+mass);
  double ha1 = 1-rho2*(w-1);

  *a1f = (1.0 - (t/((mb+mass)*(mb+mass))))*ha1;
  *a1f = (*a1f)/rstar;
  *a2f = (r2/rstar)*ha1;
  *vf = (r1/rstar)*ha1;
  *a0f = 0.0;

  return;
 }
\end{verbatim}
The {\tt getvectorff} routine takes the parent meson,
daughter vector meson, $q^2$ ({\tt t}), and the daughter meson
mass.  The form factors $A_1$, $A_2$, $V$ and $A_0$ are
returned.

\noindent The {\tt EvtHQET} class is also quite easy to
implement.  As with other classes which handle decay
models, it is derived from {\tt EvtDecayAmp}, which
is described elsewhere.  The class
should provide a {\tt Decay} routine as well as 
{\tt getName}, {\tt Init}, {\tt clone} 
and {\tt InitProbMax} routines.  Additionally,
the form factor model and amplitude calculation 
implementations are handled as memeber data.  The
{\tt EvtHQET} implementation is:
\begin{verbatim}
void EvtHQET::getName(char *model_name,int &narg){

  strcpy(model_name,"HQET");
}

EvtDecayBase* EvtHQET::clone(){

  return new EvtHQET;

}


void EvtHQET::Decay( EvtParticle *p ){

  p->init_daughters(ndaug,daug);
  calcamp->CalcAmp(p,hqetffmodel);
}

void EvtHQET::InitProbMax() {

  SetProbMax(20000.0);
}

void EvtHQET::Init(){

  if ( ndaug!=3 ) {
     report(ERROR,"EvtGen") << "Wrong number of daughters in EvtHQET.cc\n";
  }

  hqetffmodel = new EvtHQETFF(args[0],args[1],args[2]);
  calcamp = new EvtSemiLeptonicVectorAmp;
}
\end{verbatim}
This class would be virtually unchanged for different
form factor models.  In {\tt Decay}, after initialization
of the daughter particles, {\tt CalcAmp} (in this case
from {\tt EvtSemiLeptonicVectorAmp} is called.  If the
standard form factor parameterizations are used, this
routine does not need to be rewriten.  The {\tt Init}
function is responsible for two things.  First the correct
form factor model must be initialized.  As {\tt EvtHQETFF}
takes arguements, these must be passed into
the constructor.  The arguements are otherwise private to the
{\tt EvtHQET} model.  Second, the {\tt CalcAmp} function
must be properly initialized such that the amplitude
is calculated for vector daughter mesons.  For models with form
factors for daughter mesons of different spins, the meson
spin ({\tt EvtPDL::spin(daug[0])} should be used to decide 
how to initialize {\tt calcamp}.

\noindent With the existing framework, the implementation of
new form factor models should be quite simple.  The majority of
the work is in the coding of the form factors themselves

\subsection{Available models}
\label{sec:models}
\noindent There are several semileptonic form factor models
implemented within EvtGen and the {\tt EvtSemiLeptonicAmp} 
framework.  In this section we describe the basic features of
each, including the modes available in each model and what
arguements must be passed in.  This is by no means a complete
list of all possible semileptonic models, and the authors would
welcome any additional form factor models implemented.

\subsubsection{EvtSLPole}
\label{sect:EvtSLPole}
\index{EvtSLPole}
\begin{itemize} 
\item A general pole form is implemented for each form factor, using
the form
\begin{equation}
F = { {F(0)} \over { (1.0 + a { {q^2} \over {M^2} } + b { {q^4} \over {M^4} })^p}}
\end{equation}
where $M$ is the mass of the parent meson. 
\item Leptons available: $e$, $\mu$, and $\tau$.
\item Arguements: For each form factor there are four arguements, $F(0)$,
$a$, $b$, and $p$.  Note that $p$ can be a non-integer.  
\begin{itemize}
\item For scalar daughters the arguements should be ordered
as \\
$F(0)_{f_+}~ a_{f_+}~ b_{f_+}~ p_{f_+}~ F(0)_{f_0}~ a_{f_0}~ b_{f_0}p_{f_0}$.
\item For vector daughters the arguements should be ordered
as \\
$F(0)_{A_1}~ a_{A_1}~ b_{A_1}~p_{A_1}~F(0)_{A_2}~ a_{A_2}~ b_{A_2}~p_{A_2}~
F(0)_{V}~ a_{V}~ b_{V}~ p_{V}~
F(0)_{A_0}~ a_{A_0}~ b_{A_0}~p_{A_0}$
\item For tensor daughters the arguements should be ordered
as  \\ 
$F(0)_{k}~ a_{k}~ b_{k}~p_{k}~F(0)_{h}~ a_{h}~ b_{h}~p_{h}~F(0)_{b_+}~ a_{b_+}~ b_{b_+}~
p_{b_+}~F(0)_{b_-}~ a_{b_-}~ b_{b_-}~p_{b_-}$
\end{itemize}
\item In DECAY.DEC
\begin{verbatim}
0.0210   D+     e-   anti-nu_e          SLPOLE .... arguements;
\end{verbatim}
\end{itemize}

\subsubsection{EvtISGW2}
\index{EvtISGW2}
\index{ISGW2}
\begin{itemize}
\item Implementation of form factors defined from
Scora and Isgur~\cite{Scora95}.
\item Leptons available: $e$, $\mu$, and $\tau$.
\item Arguements: None
\item In DECAY.DEC
\begin{verbatim}
0.0210   D+     e-   anti-nu_e          ISGW2;
\end{verbatim}
\end{itemize}
\noindent We have implemented decays of $B$,$D$,$D_s$, and $B_s$
mesons as described by Scora and Isgur.


\subsubsection{EvtISGW}
\begin{itemize}
\item Implementation of form factors defined from
Isgur, Scora, Grinstein, and Wise~\cite{Isgur89a}.
\item Leptons available: $e$ and $\mu$ only.
\item Arguements: None
\item In DECAY.DEC
\begin{verbatim}
0.0210   D+     e-   anti-nu_e          ISGW;
\end{verbatim}
\end{itemize}
%\noindent Refer to Tables~\ref{table:bdecays} to~\ref{table:ddecays}
%for a list of
%available decays.

\subsubsection{EvtHQET}
\index{EvtHQET}
\begin{itemize}
\item Simple implementation of form factors within
HQET using a linear form for $\rho^2$.  This model is described
in more detail above. 
\item Leptons available: $e$ and $\mu$ only.
\item Arguements:$\rho^2$, $R1$, and $R2$. 
\item In DECAY.DEC
\begin{verbatim}
0.050   D*+     e-   anti-nu_e          HQET 0.92 1.18 0.72;
\end{verbatim}
\end{itemize}
\noindent This model is most useful for the decay $B \rightarrow D^* \ell \nu$,
however no assumptions are made about either the parent meson or the
daughter meson.  Any pseudoscalar semileptonic decay to a vector meson
could be modeled using these form factors.



%\begin{table}
%\begin{center}
%\begin{tabular}{|c|c|c|c|} \hline
%Decay                                           & ISGW2 & ISGW & HQET \\ \hline \hline
%$B \rightarrow D \ell \nu$                      & Y     & Y    &       \\
%$B \rightarrow D^* \ell \nu$                    & Y     & Y    & Y    \\
%$B \rightarrow D_1 \ell \nu$                    & Y     & Y    &      \\
%$B \rightarrow D_0^* \ell \nu$                  & Y     & Y    &      \\
%$B \rightarrow D_1^{\prime} \ell \nu$           & Y     & Y    &      \\
%$B \rightarrow D_2^* \ell \nu$                  & Y     & Y    &      \\
%$B \rightarrow D({\rm 2S}) \ell \nu$            & Y     & Y    &      \\
%$B \rightarrow D^*({\rm 2S}) \ell \nu$          & Y     & Y    &      \\
%$B \rightarrow \pi \ell \nu$                    & Y     & Y    &      \\
%$B \rightarrow \eta^{(\prime)} \ell \nu$        & Y     & Y    &      \\
%$B \rightarrow \rho (\omega) \ell \nu$          & Y     & Y    &      \\
%$B \rightarrow a_0 (f_0^{(\prime)})\ell \nu$    & Y     & Y    &      \\
%$B \rightarrow a_1 (f_1^{(\prime)})\ell \nu$    & Y     & Y    &      \\
%$B \rightarrow a_2 (f_2^{(\prime)})\ell \nu$    & Y     & Y    &      \\
%$B \rightarrow b_1 (h_1^{(\prime)}) \ell \nu$   & Y     & Y    &      \\
%$B \rightarrow \pi({\rm2S}) (\eta) \ell \nu$    & Y     & Y    &      \\
%$B \rightarrow \rho({\rm2S}) (\omega) \ell \nu$ & Y     & Y    &      \\ \hline
%\end{tabular}
%\end{center}
%\caption{List of available $B$ semileptonic decays for
%each form factor model currently implemented in EvtGen.  Isospin
%partners are usually implied.  Some models may not include
%form factors to properly handle decays to $\tau$ leptons.  
%An ``Y'' is given to a decay if that model was implemented with
%that particular decay in mind.  For example the HQET model is
%meant primarily for $B \rightarrow D^* \ell \nu$, however any
%pseudoscalar decay to a vector meson could be handled with the
%proper constants passed in from DECAY.DEC.}
%\label{table:bdecays} 
%\end{table}



%\begin{table}
%\begin{center}
%\begin{tabular}{|c|c|c|c|} \hline
%Decay                                           & ISGW2 & ISGW & HQET \\ \hline \hline
%\end{tabular}
%\end{center}
%\caption{List of available $D$ semileptonic decays for
%each form factor model currently implemented in EvtGen.  Isospin
%partners are usually implied.}
%\label{table:ddecays} 
%\end{table}




\end{document}




